\documentclass[11pt, a4paper, onecolumn, twoside]{report}
\usepackage{polski}
\usepackage[utf8]{inputenc}
\usepackage{fancyhdr}
\usepackage{graphicx}

\title{Środowisko programisty}
\author{Zlata Yankovska}
\date{\today}

\pagestyle{fancy}
\fancyhead[R]{cos}
\fancyfoot[R]{}

\begin{document}

\begin{titlepage}
\maketitle
\end{titlepage}

\tableofcontents

\chapter{Linux}

\Large Tematy: \\
\begin{itemize}
\item Podstawowe polecenia systemu Linux;
\item Edytory textowe: vi, emacs;
\item Skrypty Bash;
\end{itemize}

\section{Podstawowe polecenia}
\Large Powłoka\\
\normalsize Powłoka jest interpreterem poleceń.\\
\textbf{BASH} (\textit{ang. Bourne Again SHell}) to najbardziej popularna powłoka używana na systemach unixowych, jest też skryptowym językiem programowania.~\cite{bash}\\

\textit{Węcej o Bashu w  \ref{sec:Bash}} \\
\newpage
W \ref{tab} jest przedstawiono polecenia \\
%"prawa" można znaleźć w tablicy \ref{tab:Prawa}
"prawa" można znaleźć w tablicy 1.2 \\
\begin{table}
\begin{tabular}{||c||c||}
\hline\hline
\Large Polecenie & \Large Komenda\\\hline\hline
&\\
Przerwanie procesu & \textbf{CTRL-C} \\
Login urzytkownika & \textbf{who}/\textbf{whoami} \\
Nazwa bierzącego katalogu & \textbf{pwd} \\
Zmiana katalogu & \textbf{cd} \\
Zawartość katalogu & \textbf{ls} \\
Tworzenie pliku & \textbf{touch \textit{"nazwa"}} \\
Tworzenie katalogu & \textbf{mkdir \textit{"nazwa"}} \\
Usuwanie pliku & \textbf{rm \textit{"nazwa"}} \\
Usuwanie katalogu & \textbf{rmdir \textit{"nazwa"}} \\
Kopiowanie pliku & \textbf{cp \textit{"nazwa pliku"}\textit{"nazwa nowego pliku"}} \\
Kopiowanie katalogu & \textbf{cp -R \textit{"nazwa katalogu"}\textit{"nazwa nowego katalogu"}}\\
Zmiana nazwy & \textbf{mv \textit{"stara nazwa"}\textit{"nowa nazwa"}} \\
Przeniesienie pliku & \textbf{mv \textit{"ścieżka do pliku"}\textit{"nowa ścieżka do pliku"}} \\
Zmiana praw dostępu & \textbf{chmod} "prawa"\textbf{\textit{"nazwa pliku"}} \\
 & \\\hline\hline
\end{tabular}
\label{tab}
\caption{Polecenia}
\end{table}

\begin{table}
\begin{tabular}{c|c}
Znak & Wykonanie \\\hline
+ & \textbf{dodanie prawa}\\
- &  \textbf{odebranie prawa}\\
= & \textbf{ustalenie całego zestawu}\\
\hline
u & \textbf{właściciel} (=user)\\
g & \textbf{grupa} (=group)\\
o & \textbf{pozostali} (=other)\\
\hline
r=4 & \textbf{prawo do czytania} (=read)\\
w=2 & \textbf{prawo do pisania} (=write)\\
x=1 & \textbf{prawo do wykonywania} (=execute)\\
\hline
\end{tabular}
%\label{tab:Prawa}
\caption{Prawa dostępu}
\end{table}

\Large Przykłady zmiany praw dostępu \\
\normalsize
\textbf{chmod u+x plik1} \textit{- nadanie użytkownikowi prawa wykonywania}\\
\textbf{chmod o-r plik1} \textit{- odebranie pozostałym prawa czytania}\\
\textbf{chmod go-rwx plik1} \textit{- odebranie grupie i innym wszystkich praw}\\
\textbf{chmod 604 plik1} \textit{= użytkownik ma prawa do czytania i pisania; grupy nie mają żadnych praw; pozostali mają prawa do czytania~}

\section{Edytory textowe}
\Large VI\\
\normalsize  vi – jest mały; jest częścią środowiska uniksowego\\
\begin{itemize}
\item Vi to wyłącznie edytor tekstu
\item W vi wiele istotnych poleceń jest przypisanych pojedynczym klawiszom
\item W trakcie pracy dokonuje się przełączania między trybem, w którym wpisanie znaków powoduje wstawienie ich do redagowanego tekstu, a trybem, w którym te same znaki są poleceniami, np. klawisze h, j, k i l służą jako strzałki do poruszania się po tekście. ~
\item Vi również identyfikuje typ pliku i przechodzi do odpowiedniego trybu, w którym rozpoznaje składnię języka i pomaga rozróżnić poszczególne słowa.~ \\
\textit{Materiały wziąte z \cite{Pani}}
\end{itemize}

\section{Programowanie w Bashu}
\label{sec:Bash}
\Large Skrypt
\normalsize 
\begin{itemize}
\item Skrypt to niekompilowany tekstowy plik wykonywalny, zawierający jakieś polecenia systemowe oraz polecenia sterujące jego wykonaniem (instrukcje, pętle itp.) ~
\item Skrypt wykonywany jest tylko i wyłącznie przez interpreter (tutaj /bin/bash), który tłumaczy polecenia zawarte w skrypcie na język zrozumiały dla procesora ~ \\
\textit{Materiały wziąte z \cite{Pani}}
\end{itemize}
\Large Pierwszy program \\
\normalsize Sprobujemy napisać swój pierwszy skrypt, który wypisze "Hello, world". ~ \\
\\
\emph {\Large !/bin/bash \\
echo "Hello world"} \\

\chapter{\LaTeX}
\section{Tworzenie dokumentów}
\Large Przykład
\normalsize 
\textbf{\\documentclass}[a4paper,12pt]\{article\}
\textbf{\\usepackage}[latin2]\{inputenc\}
\textbf{\\usepackage}[T1]\{fontenc\} 
\textbf{\\usepackage}\{times\}
\textbf{\\title}\{Tytuł\}
\textbf{\\author}\{Autor\}
\textbf{\\begin}\{document\} 
\textbf{\\maketitle}
\textbf{\\noindent }
\\\large Jakaś treść \normalsize
\textbf {\\end}\{document\} \\\\
\textit{Materiały wziąte z \cite{Pani}}


\section{Wyrażenia matematyczne}
\Large Przykłady wyrażeń matematycznych

$$a\equiv_{11} b$$
\\
$$\underbrace{1+1+\overbrace{1+\ldots +1}\limits^{10}}\limits_{20}$$
\\
$$\bigcup\limits^{n}_{s=1} A_s = A_1 \cup A_2 \cup \ldots \cup A_n$$
\\
$$\frac{\frac{1}{x+y}-1}{a+b+c}$$
\\
$$\sum\limits^{\infty}_{n=1} \left( \left\lfloor \frac{1}{\sqrt{n}+2} \right\rfloor \right)$$
\\
$${n \choose k}= {n-1 \choose k-1} + {n-1 \choose k}$$
\\
$$\prod\limits^{n+1}_{i=1} a_i = \prod\limits^{n}_{i=1} a_i \cdot a_{i+1}$$
\normalsize
\\
\begin{figure}[ht]
\begin{center}
\includegraphics {/home/LABPK/zyankovska/Obrazy/latex.png}
\end{center}
\end{figure}

\begin{thebibliography}{9}
\bibitem{bash}
C. Albing; JP Vossen; C. Newham, \textit{Bash receptury}, Helion, 2008.
\bibitem{[2]}
T. Oetiker; H. Partle; I. Hyna; E. Schlegl, \textit{Nie za krótkie wprowadzenie do systemu LaTeX2e}, 2006.
\bibitem{[3]}
A.Robbins; N. H. F. Beebe, \textit{Programowanie skryptów powloki}, Helion, 2005.
\bibitem{Pani} Prezentacji z wykładu Pani E.Lubeckiej :)
\end{thebibliography}
\end{document}